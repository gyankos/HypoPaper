\section{Hypothesis  Query Answering}\label{sec:synthansw}
As previously stated, each ``minimal'' connected (hyper)graph containing the candidates describes an hypothesis $h$. Now, we want to align $h$ towards the query representation: such alignment can be represented by instantiating the query variables in $q$ with the candidates, thus obtaining the to-be-returned hypothesis $h'$. Given that $h$ was obtained through approximate graph matching of $q$, we know that the greater the edit distance between the two $h$ and $h'$, the less the reliability on $h'$ as a possible candidate. By doing so we assign higher scores to the hypotheses that are directly represented within the data, and inferior ones to all the remaining hypotheses that were generated over imprecise(?) inference rules. The combination of the support measure with the edit distance provides the score assigned to $h'$.