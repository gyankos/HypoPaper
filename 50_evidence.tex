\section{Hypothesis Evidence and Scoring}\label{sec:scoring}
In contrast to the evidence in the former section, having coherent candidates does not prevent us from having hypotheses that either contain contradictions or are contradicted by other facts or relationships. It is then required to associate each generated hypothesis with the set of facts and relationships providing either evidence or counterevidence. 

Let us analyse some possible inconsistencies: if we suppose that the relationship ``\textit{Abigail has never been in the USA}'' is stored within our knowledge base, then we now that this evidence contradicts the facts in Table \ref{tab:datahyp} generating the candidates ``\textit{Minneapolis}'' and ``\textit{Duluth}''. On the other hand, the relationship ``\textit{Abigail was born in Detroit on 1989}''  contradicts with the second ``\textit{Rome}'' candidate, because the entity \textit{Abigail} did not existed prior to 1989. Last, the fact ``\textit{On the 22\textsuperscript{nd} of June 2018, Abigail died in Turin}'' contradicts with the hypothesis ``\textit{Latium}'' because the same person cannot be located in two different places at the same time. 

The former inconsistencies may be observed by comparing different KB facts and relationships with the hypothesis, which combination describes a subset of the whole knowledge graph, $\kappa$. We can now represent facts and tuples from $\kappa$ as tuples, thus allowing to use logical inconsistent detection frameworks which are independent from the graph representation of the data \cite{MINOUX19881}. Rules may be directly provided by the ontologies' TBox-es \texttt{[cite:TODO]}, which describe the correlations between different facts and relationships and provide an homogeneous representation of different facts and relationships. E.g., the TBox may state that each \texttt{movement} fact towards a given destination at a given time implies a \texttt{location} relationship of the same subject at the same time. After converting each possible fact and relationships into ``finer-grained'' facts and relationships, it is now possible to recognize all the spatio-temporal inconsistencies as in \cite{gmmp18}. This approach can  be therefore extended to any other type of inconsistencies foreseen by the  TBox rules in ontologies. 

Last, we can now measure its discrepancies via support scores, thus providing a first ranking of the generated hypotheses. 